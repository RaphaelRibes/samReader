%! Author = raphael
%! Date = 12/10/24

% Preamble
\documentclass[11pt]{article}

\usepackage[french]{babel}

\usepackage[style=numeric, sorting=none, backend=biber]{biblatex}
\addbibresource{samReader.bib}

% Packages
\usepackage{float}
\usepackage{makecell} % Pour permettre les cellules sur plusieurs lignes
\usepackage{amsmath}
\usepackage[bottom=25mm, top=25mm]{geometry}
\usepackage{graphicx}
\usepackage{titlesec}
\usepackage{booktabs} % Pour des tableaux plus esthétiques


\usepackage[hidelinks]{hyperref}

\usepackage{caption}
\captionsetup[table]{name=Tableau}
\captionsetup[figure]{name=Figure}

% Document
\begin{document}

\begin{titlepage}
    \begin{center}
        % Logos
        \includegraphics[width=0.4\textwidth]{UM} \hfill{} % Replace with your first logo
        \includegraphics[width=0.4\textwidth]{FDS} % Replace with your third logo

        \vspace{3cm}

        % Title with bars
        \hrulefill\\[0.4cm]
        {\Huge \textbf{Projet Bioinformatique : samReader}}\\[0.4cm]
        \hrulefill

        \vspace{1.5cm}

        \large{\textbf{Raphaël Ribes}}\\[0.5cm]

        \vfill

        % Horizontal image near the bottom
        \url{https://github.com/RaphaelRibes/samReader}
        \href{https://github.com/RaphaelRibes/samReader}{%
            \includegraphics[width=0.8\textwidth]{samReader} % Replace with your horizontal image
        }
    \end{center}
\end{titlepage}

\section{Introduction}\label{sec:introduction}
Depuis l'émergence des technologies de séquençage de nouvelle génération (NGS), le volume et la complexité des données de séquences ADN ont augmenté de manière exponentielle.
Ces technologies, bien que révolutionnaires, génèrent des courts fragments de séquences ADN, appelés reads, qu'il est nécessaire d'aligner sur une séquence de référence pour les analyser efficacement.
Afin de standardiser et de faciliter le stockage et la manipulation de ces données d'alignement, le format SAM (Sequence Alignment/Map) a été introduit par Heng Li et Bob Handsaker \textit{et al.} en 2009\cite{li_sequence_2009}.

\noindent Le format SAM offre une solution simple et flexible pour organiser les résultats d'alignement grâce à un fichier tabulé structuré, comprenant une section d'en-tête et une section de données d'alignement.
Ce format, ainsi que son grand frère le format BAM, sont devenus indispensable pour des applications variées en biologie plus particulièrement en sciences omique.
\\\\
Ce Programme vise à lire, analyser et extraire les informations essentielles des fichiers SAM\@.
Le but est de faciliter l'interprétation des alignements, leur profondeur et leur qualité.
Cet outil permet une analyse des données issues des technologies de NGS plus simple et compréhensible pour les chercheurs.


\section{Présentation du format SAM}\label{sec:presentation-du-format-sam}
Le format SAM comporte une section d'en-tête optionnelle, constituée de lignes commençant par le caractère \texttt{@}.
Chaque ligne représente un enregistrement spécifique grâce à un code à deux lettres, suivi de champs sous la forme \texttt{TAG:VALUE}.
Les en-têtes courants incluent \texttt{@HD} pour les informations générales du fichier comme la version et l'ordre de tri, \texttt{@SQ} pour décrire les séquences de référence (nom, longueur, etc.), \texttt{@RG} pour les groupes de lectures (identifiant, échantillon, plateforme), et \texttt{@PG} pour tracer les programmes ayant traité les alignements.
Enfin, les lignes \texttt{@CO} permettent d'ajouter des commentaires libres.
Ces en-têtes jouent un rôle essentiel dans l'organisation et l'interprétation des données d'alignement.
\\\\
Dans le format SAM, chaque ligne d'alignement représente typiquement l'alignement linéaire d'un segment.
Chaque ligne est composée de 11 champs ou plus, séparés par des tabulations.
Les onze premiers champs sont toujours présents et dans l'ordre suivant ; si l'information représentée par l'un de ces champs est indisponible, une valeur de substitution est utilisée : soit \texttt{'0'} soit \texttt{'*'}, en fonction du type du champ.
Le tableau suivant (\autoref{tab:mandatory-fields}) donne un aperçu de ces champs obligatoires dans le format SAM :

\begin{table}[H]
    \centering
    \begin{tabular}{|c|c|p{9cm}|}
        \hline
        \textbf{Col} & \textbf{Champ} & \textbf{Description succincte} \\
        \hline
        1 & QNAME & Nom de la requête ou du modèle (chromosome) \\
        2 & FLAG & Indicateurs binaires \\
        3 & RNAME & Nom de la séquence de référence \\
        4 & POS & Position de mappage la plus à gauche (indexée à 1) \\
        5 & MAPQ & Qualité du mappage \\
        6 & CIGAR & Chaîne CIGAR \\
        7 & RNEXT & Nom de la séquence de référence du segment suivant \\
        8 & PNEXT & Position du segment suivant \\
        9 & TLEN & Longueur observée du modèle \\
        10 & SEQ & Séquence du segment \\
        11 & QUAL & Qualité des bases \\
        \hline
    \end{tabular}
    \caption{Champs obligatoires dans une ligne d'alignement SAM.}
    \label{tab:mandatory-fields}
\end{table}

\begin{itemize}
    \item \textbf{QNAME :} Nom de la requête ou du modèle.
    Les segments ayant le même QNAME sont supposés provenir du même modèle.
    Une valeur \texttt{'*'} indique que l'information est indisponible.
    Dans un fichier SAM, une lecture peut occuper plusieurs lignes d'alignement si son alignement est chimérique ou s'il existe plusieurs mappages.

    \item \textbf{FLAG :} Indicateurs combinés sous forme binaire.
    Chaque bit a une signification particulière décrit dans le tableau suivant (\autoref{tab:bit-values}).
    \begin{table}[h!]
        \centering
        \begin{tabular}{lp{10.3cm}}
            \toprule
            \textbf{Bit} & \textbf{Description} \\
            \midrule
            1    & Modèle ayant plusieurs segments dans le séquençage \\
            2    & Chaque segment correctement aligné selon l'outil d'alignement \\
            4    & Segment non aligné \\
            8    & Segment suivant dans le modèle non aligné \\
            16   & SEQ est inversée complémentaire \\
            32   & SEQ du segment suivant est inversée complémentaire \\
            64   & Premier segment dans le modèle \\
            128  & Dernier segment dans le modèle \\
            256  & Alignement secondaire \\
            512  & Ne passe pas les filtres (contrôles de qualité) \\
            1024 & PCR ou duplicat optique \\
            2048 & Alignement supplémentaire \\
            \bottomrule
        \end{tabular}
        \caption{Description des valeurs de bits utilisées pour les alignements.}
        \label{tab:bit-values}
    \end{table}

    \noindent Par exemple \texttt{163} convertit en binaire donne \texttt{000010100011} ce qui signifie que le segment est aligné, est le dernier segment dans le modèle et que le prochain segment est aligné et inversé complémenté.

    \item \textbf{RNAME :} Nom de la séquence de référence de l'alignement.
    Si des lignes d'en-tête @SQ sont présentes, RNAME (si différent de \texttt{'*'}) doit être défini dans l'un des tags \texttt{SN} des lignes \texttt{@SQ}.

    \item \textbf{POS :} Position de mappage la plus à gauche, indexée à 1, de la première opération CIGAR qui "consomme" une base de la séquence de référence.

    \item \textbf{MAPQ :} Qualité du mappage, calculée comme $-10 \log_{10} Pr\{\text{la position de mappage est erronée}\}$, arrondie à l'entier le plus proche.
    Une valeur de 255 indique que la qualité n'est pas disponible.

    \item \textbf{CIGAR :} Chaîne CIGAR qui décrit l'alignement entre la séquence de la requête et la séquence de référence.
    Chaque opération est représentée par un chiffre suivi d'une lettre\autoref{tab:operations_alignement}.
    \begin{table}[h!]
        \centering
        \begin{tabular}{clcc}
        \hline
        \textbf{Op} & \textbf{Description} &
        \makecell{\textbf{Consomme}\\\textbf{la requête}} &
        \makecell{\textbf{Consomme}\\\textbf{la référence}} \\ \hline
        M & Correspondance/Discordance d'alignement & oui & oui \\
        I & Insertion dans la référence & oui & non \\
        D & Suppression de la référence & non & oui \\
        N & Région ignorée dans la référence & non & oui \\
        S & Recouvrement souple (séquences coupées présentes dans SEQ) & oui & non \\
        H & Recouvrement dur (séquences coupées absentes dans SEQ) & non & non \\
        P & Remplissage (suppression silencieuse de la référence remplie) & non & non \\
        = & Correspondance de séquence & oui & oui \\
        X & Discordance de séquence & oui & oui \\ \hline
        \end{tabular}
        \caption{Description des opérations d'alignement}
        \label{tab:operations_alignement}
    \end{table}
    "Consomme la requête" et "Consomme la référence" indiquent si l'opération CIGAR provoque l'avancement de l'alignement respectivement le long de la séquence de requête et de la séquence de référence.
    INSÈRE UNE IMAGE AVEC DELETION, MAPPAGE ET INSERTION

    \item \textbf{RNEXT et PNEXT :} Indiquent respectivement le nom et la position de la séquence de référence du segment suivant dans le modèle.

    \item \textbf{TLEN :} Longueur observée du modèle.
    Positive pour le segment le plus à gauche, négative pour le segment le plus à droite.

    \item \textbf{SEQ :} Séquence du segment.
    Peut être \texttt{'*'} si non stockée.

    \item \textbf{QUAL :} Qualité des bases, encodée en ASCII avec un décalage de 33.
\end{itemize}

\section{Exemple d'application}\label{sec:exemple-d'application}


\section{Description du programme}\label{sec:description-du-programme}

Le programme commence en premier avec le script \texttt{samReader.sh}, qui avant de lancer le programme, réalise plusieurs étapes préliminaires :
\begin{itemize}
\item La lecture et validation des options et arguments en ligne de commande.
\item La vérification de la sélection d'une version de SAM.
\item Si l’option --trusted est activée, le fichier est directement traité sans validation supplémentaire.
\item La vérification du fichier d’entrée (existence, non-vide, conformité au format SAM).
\end{itemize}

\noindent Ensuite, le script \texttt{main.py} est exécuté.
Ce dernier va itérer sur chaque ligne du fichier SAM, et vérifier (si l’option --trusted est désactivée) la conformité de chaque ligne au format SAM.
Il va aussi lors de cette étape, convertir les flags en binaire et récupérer la position du dernier read ainsi que son CIGAR.

Le programme va ensuite créer un répertoire \texttt{temp} dans lequel il va stocker les fichiers temporaires nécessaires à la génération des rapports.


\noindent Si tout est correct, exécution du script Python main.py avec les arguments adaptés.


\section{Discussion}\label{sec:discussion}

\texttt{samReader} présente des points forts significatifs, notamment la capacité à générer des rapports détaillés et personnalisés.
Ces rapports, produits sous forme de fichiers PDF, synthétisent les analyses de manière claire et exhaustive.
L’analyse approfondie des mutations fournit des statistiques générales sur les anomalies présentes dans les reads.
De plus, la segmentation des données par chromosomes améliore la lisibilité en regroupant les reads alignés, partiellement alignés et non alignés dans des répertoires distincts.
L’évaluation de la profondeur de couverture et la représentation de l’évolution de la qualité d’alignement sont aussi des fonctionnalités précieuses pour les chercheurs.


Cependant, plusieurs limitations freinent l’efficacité du programme.
La lenteur de son exécution constitue un obstacle majeur, en grande partie en raison de la génération des rapports et des graphiques, particulièrement via \LaTeX.
La représentation base par base de la qualité et de la profondeur de mappage est également problématique, limitant la précision des analyses.
Par ailleurs, le traitement des données de qualité de mappage n’est pas optimal : lorsque deux reads se superposent, l’information précédente est écrasée, ce qui peut entraîner une perte de données essentielles.


Pour améliorer ces aspects, des pistes d’optimisation sont envisageables.
La vitesse d’exécution pourrait être augmentée par une relecture plus efficace des reads et une \\ implémentation dynamique des graphiques.
Une meilleure gestion des données MAPQ est également recommandée, en stockant séparément les valeurs pour chaque base et en calculant ultérieurement une moyenne ou une médiane.
L'implémentation des différentes versions de SAM reste encore perfectible.
Bien que les idées soient présentes, leur réalisation nécessite des ajustements et des améliorations pour atteindre une architecture satisfaisante.
Ces ajustements permettraient d’accroître la précision et la rapidité du programme tout en conservant sa flexibilité.

\addcontentsline{toc}{section}{References}
\printbibliography

\end{document}