Le format SAM comporte une section d'en-tête optionnelle, constituée de lignes commençant par le caractère \texttt{@}.
Chaque ligne représente un enregistrement spécifique grâce à un code à deux lettres, suivi de champs sous la forme \texttt{TAG:VALUE}.
Les en-têtes courants incluent \texttt{@HD} pour les informations générales du fichier comme la version et l'ordre de tri, \texttt{@SQ} pour décrire les séquences de référence (nom, longueur, etc.), \texttt{@RG} pour les groupes de lectures (identifiant, échantillon, plateforme), et \texttt{@PG} pour tracer les programmes ayant traité les alignements.
Enfin, les lignes \texttt{@CO} permettent d'ajouter des commentaires libres.
Ces en-têtes jouent un rôle essentiel dans l'organisation et l'interprétation des données d'alignement.
\\\\
Dans le format SAM, chaque ligne d'alignement représente typiquement l'alignement linéaire d'un segment.
Chaque ligne est composée de 11 champs ou plus, séparés par des tabulations.
Les onze premiers champs sont toujours présents et dans l'ordre suivant ; si l'information représentée par l'un de ces champs est indisponible, une valeur de substitution est utilisée : soit \texttt{'0'} soit \texttt{'*'}, en fonction du type du champ.
Le tableau suivant (\autoref{tab:mandatory-fields}) donne un aperçu de ces champs obligatoires dans le format SAM :

\begin{table}[H]
    \centering
    \begin{tabular}{|c|c|p{9cm}|}
        \hline
        \textbf{Col} & \textbf{Champ} & \textbf{Description succincte} \\
        \hline
        1 & QNAME & Nom de la requête ou du modèle (chromosome) \\
        2 & FLAG & Indicateurs binaires \\
        3 & RNAME & Nom de la séquence de référence \\
        4 & POS & Position de mappage la plus à gauche (indexée à 1) \\
        5 & MAPQ & Qualité du mappage \\
        6 & CIGAR & Chaîne CIGAR \\
        7 & RNEXT & Nom de la séquence de référence du segment suivant \\
        8 & PNEXT & Position du segment suivant \\
        9 & TLEN & Longueur observée du modèle \\
        10 & SEQ & Séquence du segment \\
        11 & QUAL & Qualité des bases \\
        \hline
    \end{tabular}
    \caption{Champs obligatoires dans une ligne d'alignement SAM.}
    \label{tab:mandatory-fields}
\end{table}

\begin{itemize}
    \item \textbf{QNAME :} Nom de la requête ou du modèle.
    Les segments ayant le même QNAME sont supposés provenir du même modèle.
    Une valeur \texttt{'*'} indique que l'information est indisponible.
    Dans un fichier SAM, une lecture peut occuper plusieurs lignes d'alignement si son alignement est chimérique ou s'il existe plusieurs mappages.

    \item \textbf{FLAG :} Indicateurs combinés sous forme binaire.
    Chaque bit a une signification particulière décrit dans le tableau suivant (\autoref{tab:bit-values}).
    \begin{table}[h!]
        \centering
        \begin{tabular}{lp{10.3cm}}
            \toprule
            \textbf{Bit} & \textbf{Description} \\
            \midrule
            1    & Modèle ayant plusieurs segments dans le séquençage \\
            2    & Chaque segment correctement aligné selon l'outil d'alignement \\
            4    & Segment non aligné \\
            8    & Segment suivant dans le modèle non aligné \\
            16   & SEQ est inversée complémentaire \\
            32   & SEQ du segment suivant est inversée complémentaire \\
            64   & Premier segment dans le modèle \\
            128  & Dernier segment dans le modèle \\
            256  & Alignement secondaire \\
            512  & Ne passe pas les filtres (contrôles de qualité) \\
            1024 & PCR ou duplicat optique \\
            2048 & Alignement supplémentaire \\
            \bottomrule
        \end{tabular}
        \caption{Description des valeurs de bits utilisées pour les alignements.}
        \label{tab:bit-values}
    \end{table}

    \noindent Par exemple \texttt{163} convertit en binaire donne \texttt{000010100011} ce qui signifie que le segment est aligné, est le dernier segment dans le modèle et que le prochain segment est aligné et inversé complémenté.

    \item \textbf{RNAME :} Nom de la séquence de référence de l'alignement.
    Si des lignes d'en-tête @SQ sont présentes, RNAME (si différent de \texttt{'*'}) doit être défini dans l'un des tags \texttt{SN} des lignes \texttt{@SQ}.

    \item \textbf{POS :} Position de mappage la plus à gauche, indexée à 1, de la première opération CIGAR qui "consomme" une base de la séquence de référence.

    \item \textbf{MAPQ :} Qualité du mappage, calculée comme $-10 \log_{10} Pr\{\text{la position de mappage est erronée}\}$, arrondie à l'entier le plus proche.
    Une valeur de 255 indique que la qualité n'est pas disponible.

    \item \textbf{CIGAR :} Chaîne CIGAR qui décrit l'alignement entre la séquence de la requête et la séquence de référence.
    Chaque opération est représentée par un chiffre suivi d'une lettre\autoref{tab:operations_alignement}.
    \begin{table}[h!]
        \centering
        \begin{tabular}{clcc}
        \hline
        \textbf{Op} & \textbf{Description} &
        \makecell{\textbf{Consomme}\\\textbf{la requête}} &
        \makecell{\textbf{Consomme}\\\textbf{la référence}} \\ \hline
        M & Correspondance/Discordance d'alignement & oui & oui \\
        I & Insertion dans la référence & oui & non \\
        D & Suppression de la référence & non & oui \\
        N & Région ignorée dans la référence & non & oui \\
        S & Recouvrement souple (séquences coupées présentes dans SEQ) & oui & non \\
        H & Recouvrement dur (séquences coupées absentes dans SEQ) & non & non \\
        P & Remplissage (suppression silencieuse de la référence remplie) & non & non \\
        = & Correspondance de séquence & oui & oui \\
        X & Discordance de séquence & oui & oui \\ \hline
        \end{tabular}
        \caption{Description des opérations d'alignement}
        \label{tab:operations_alignement}
    \end{table}
    "Consomme la requête" et "Consomme la référence" indiquent si l'opération CIGAR provoque l'avancement de l'alignement respectivement le long de la séquence de requête et de la séquence de référence.
    INSÈRE UNE IMAGE AVEC DELETION, MAPPAGE ET INSERTION

    \item \textbf{RNEXT et PNEXT :} Indiquent respectivement le nom et la position de la séquence de référence du segment suivant dans le modèle.

    \item \textbf{TLEN :} Longueur observée du modèle.
    Positive pour le segment le plus à gauche, négative pour le segment le plus à droite.

    \item \textbf{SEQ :} Séquence du segment.
    Peut être \texttt{'*'} si non stockée.

    \item \textbf{QUAL :} Qualité des bases, encodée en ASCII avec un décalage de 33.
\end{itemize}